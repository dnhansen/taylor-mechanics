% Document setup
\documentclass[article, a4paper, 11pt, oneside]{memoir}
\usepackage[utf8]{inputenc}
\usepackage[T1]{fontenc}
\usepackage[UKenglish]{babel}

% Document info
\newcommand\doctitle{Taylor, \emph{Classical Mechanics}}
\newcommand\docauthor{Danny Nygård Hansen}

% Formatting and layout
\usepackage[autostyle]{csquotes}
\usepackage[final]{microtype}
\usepackage{xcolor}
\frenchspacing
\usepackage{latex-sty/articlepagestyle}
\usepackage{latex-sty/articlesectionstyle}

% Fonts
\usepackage{amssymb}
\usepackage[largesmallcaps,partialup]{kpfonts}
\DeclareSymbolFontAlphabet{\mathrm}{operators} % https://tex.stackexchange.com/questions/40874/kpfonts-siunitx-and-math-alphabets
\linespread{1.06}
% \let\mathfrak\undefined
% \usepackage{eufrak}
\DeclareMathAlphabet\mathfrak{U}{euf}{m}{n}
\SetMathAlphabet\mathfrak{bold}{U}{euf}{b}{n}
% https://tex.stackexchange.com/questions/13815/kpfonts-with-eufrak
\usepackage{inconsolata}

% Hyperlinks
\usepackage{hyperref}
\definecolor{linkcolor}{HTML}{4f4fa3}
\hypersetup{%
	pdftitle=\doctitle,
	pdfauthor=\docauthor,
	colorlinks,
	linkcolor=linkcolor,
	citecolor=linkcolor,
	urlcolor=linkcolor,
	bookmarksnumbered=true
}

% Equation numbering
\numberwithin{equation}{chapter}

% Footnotes
\footmarkstyle{\textsuperscript{#1}\hspace{0.25em}}

% Mathematics
\usepackage{latex-sty/basicmathcommands}
\usepackage{latex-sty/framedtheorems}

% Lists
\usepackage{enumitem}
\setenumerate[0]{label=\normalfont(\arabic*)}

% Bibliography
\usepackage[backend=biber, style=authoryear, maxcitenames=2, useprefix]{biblatex}
\addbibresource{references.bib}

% Title
\title{\doctitle}
\author{\docauthor}

\newcommand{\setF}{\mathbb{F}}
\newcommand{\ev}{\mathrm{ev}}
\newcommand{\calT}{\mathcal{T}}
\newcommand{\calU}{\mathcal{U}}
\newcommand{\calB}{\mathcal{B}}
\newcommand{\calE}{\mathcal{E}}
\newcommand{\calC}{\mathcal{C}}
\newcommand{\calD}{\mathcal{D}}
\newcommand{\calF}{\mathcal{F}}
\newcommand{\calG}{\mathcal{G}}
\newcommand{\calM}{\mathcal{M}}
\newcommand{\calA}{\mathcal{A}}
\newcommand{\calP}{\mathcal{P}}
\newcommand{\calR}{\mathcal{R}}
\newcommand{\calL}{\mathcal{L}}
\newcommand{\calH}{\mathcal{H}}
\newcommand{\borel}{\mathcal{B}}
\newcommand{\measurable}{\mathcal{M}}
\newcommand{\wto}{\Rightarrow}
\DeclarePairedDelimiter{\net}{\langle}{\rangle}
\newcommand{\strucS}{\mathfrak{S}}
\DeclarePairedDelimiter{\gen}{\langle}{\rangle} % Generating set
\newcommand{\frakL}{\mathfrak{L}}


% Physics commands

\DeclarePairedDelimiter{\ket}{\lvert}{\rangle}
% \renewcommand{\vec}{\mathbf}
\newcommand{\grad}{\nabla}

\usepackage{adforn}
\newcommand\fleuronbreak{\fancybreak{\textcolor{linkcolor}{\adfhangingflatleafleft}}}


\begin{document}

\maketitle


\addtocounter{chapter}{2}
\chapter{Momentum and Angular Momentum}

\addtocounter{section}{2}
\section{The Centre of Mass}

\begin{remarkbreak}[The centre of mass and convexity]
    Consider collections of particles at positions $\vec{r}_{11}, \ldots, \vec{r}_{1k}$ and $\vec{r}_{21}, \ldots, \vec{r}_{2l}$, with masses $m_{11}, \ldots, m_{1k}$ and $m_{21}, \ldots, m_{2l}$, respectively. Let $M_1 = \sum_{i=1}^k m_{1i}$ and $M_2 = \sum_{j=1}^l m_{2j}$ be the total masses of the two collections of particles, and let
    %
    \begin{equation*}
        \vec{R}_1
            = \frac{1}{M_1} \sum_{i=1}^k m_{1i} \vec{r}_{1i}
        \quad \text{and} \quad
        \vec{R}_2
            = \frac{1}{M_2} \sum_{j=1}^l m_{2j} \vec{r}_{2j}
    \end{equation*}
    %
    be their centres of mass. The centre of mass of the whole system is then
    %
    \begin{align*}
        \vec{R}
            &= \frac{1}{M_1 + M_2} \biggl( \sum_{i=1}^k m_{1i} \vec{r}_{1i} + \sum_{j=1}^l m_{2j} \vec{r}_{2j} \biggr) \\
            &= \frac{1}{M_1 + M_2} \biggl( \frac{M_1}{M_1} \sum_{i=1}^k m_{1i} \vec{r}_{1i} + \frac{M_2}{M_2} \sum_{j=1}^l m_{2j} \vec{r}_{2j} \biggr) \\
            &= \frac{M_1 \vec{R}_1 + M_2 \vec{R}_2}{M_1 + M_2}.
    \end{align*}
    %
    This obviously extends to any finite number of collections of particles.

    In particular, it follows that the centre of mass of a finite collection of particles lies in the convex hull of their position vectors. For the above shows that $\vec{R}$ is a convex combination of $\vec{R}_1$ and $\vec{R}_2$, so this follows by induction on the number of particles. In fact, this holds for any weighted average (with non-negative weights) of points in any dimension.
\end{remarkbreak}


\section{Angular Momentum for a Single Particle}

\begin{remarkbreak}[Angular momentum in noninertial frames]
    Choose an origin $O$ that is not necessarily at rest in any inertial frame, and consider a particle with mass $m$ and position vector $\vec{r}$ relative to $O$. The angular momentum of the particle relative to $O$ is then, by definition, $\vec{l} = \vec{r} \prod \vec{p}$. Its time derivative is
    %
    \begin{equation*}
        \dot{\vec{l}}
            = \dot{\vec{r}} \prod \vec{p} + \vec{r} \prod \dot{\vec{p}}
            = \vec{r} \prod \dot{\vec{p}}.
    \end{equation*}
    %
    If $O$ is at rest in an inertial frame, then $\dot{\vec{p}}$ equals the total force $\vec{F}$ on the particle, so the above equals the torque $\vec{\Gamma}$. However, if we include in $\vec{F}$ any inertial forces resulting from measuring $\vec{r}$ in a noninertial frame and define the torque from this force, then the above still equals the torque.
\end{remarkbreak}


\section{Angular Momentum for Several Particles}

\newcommand{\CM}{\mathrm{CM}}

\begin{remarkbreak}[König's theorem for angular momenta]
    For $i = 1, \ldots, n$ consider a particle with mass $m_i$ and position vector $\vec{r}_i$ with respect to an origin $O$ at rest in an inertial frame. Furthermore, let $\vec{p}_i$ be its momentum. If $\vec{R}$ is the centre of mass of all $n$ particles, then we let $\vec{r}_i' = \vec{r} - \vec{R}$ be the relative position of the $i$th particle, and let $\vec{p}_i' = m_i \dot{\vec{r}}_i'$. Denote by $M$ the total mass of the particles, and let $\vec{P}$ be the total momentum (equivalently $\vec{P} = M \dot{\vec{R}}$). The angular momentum of the $i$th particle with respect to the centre of mass is then
    %
    \begin{align*}
        \vec{l}_i'
            &= \vec{r}_i' \prod \vec{p}_i'
             = (\vec{r}_i - \vec{R}) \prod (\vec{p}_i - m_i \dot{\vec{R}}) \\
            &= \vec{l}_i - m_i \vec{r}_i \prod \dot{\vec{R}} - \vec{R} \prod \vec{p}_i + m_i \vec{R} \prod \dot{\vec{R}}.
    \end{align*}
    %
    Therefore, the total angular momentum with respect to the centre of mass is
    %
    \begin{align*}
        \vec{L}'
            &= \sum_{i=1}^n \vec{l}_i'
             = \vec{L} - M \vec{R} \prod \dot{\vec{R}} - \vec{R} \prod \vec{P} + M \vec{R} \prod \dot{\vec{R}} \\
            &= \vec{L} - \vec{R} \prod \vec{P}
             = \vec{L} - \vec{L}_\CM,
    \end{align*}
    %
    where $\vec{L}_\CM = \vec{R} \prod \vec{P}$ is the angular momentum of the centre of mass. Hence
    %
    \begin{equation*}
        \vec{L} = \vec{L}' + \vec{L}_\CM,
    \end{equation*}
    %
    which is \emph{König's theorem} for angular momenta.
\end{remarkbreak}


\begin{remarkbreak}[Decomposition of torque]
    Next let each particle $i$ be acted upon by an external force $\vec{F}_i$, and let $\vec{F}$ be the total force on the system. If the total external torque relative to the origin $O$ is $\vec{\Gamma}$, then the external torque relative to the centre of mass is
    %
    \begin{align*}
        \vec{\Gamma}'
            &= \sum_{i=1}^n \vec{r}_i' \prod \vec{F}_i
             = \sum_{i=1}^n (\vec{r}_i - \vec{R}) \prod \vec{F}_i \\
            &= \sum_{i=1}^n \vec{\Gamma}_i - \vec{R} \prod \vec{F}
             = \vec{\Gamma} - \vec{\Gamma}_\CM,
    \end{align*}
    %
    where $\vec{\Gamma}_\CM = \vec{R} \prod \vec{F}$ is the torque on the centre of mass. That is,
    %
    \begin{equation*}
        \vec{\Gamma}
            = \vec{\Gamma}' + \vec{\Gamma}_\CM.
    \end{equation*}
    %
    Now notice that 
    %
    \begin{equation*}
        \dot{\vec{L}}_\CM
            = (\dot{\vec{R}} \prod \vec{P}) + (\vec{R} \prod \dot{\vec{P}})
            = \vec{R} \prod \vec{F}
            = \vec{\Gamma}_\CM,
    \end{equation*}
    %
    since $\dot{\vec{R}}$ and $\vec{P} = M\dot{\vec{R}}$ are parallel. König's theorem thus implies that
    %
    \begin{equation*}
        \dot{\vec{L}}'
            = \dot{\vec{L}} - \dot{\vec{L}}_\CM
            = \vec{\Gamma} - \vec{\Gamma}_\CM
            = \vec{\Gamma}'.
    \end{equation*}
    %
    That is, relative to the centre of mass, the external torque is the time derivative of the angular momentum, even if the centre of mass frame is not inertial.
\end{remarkbreak}


\chapter{Energy}

\section{Kinetic Energy and Work}

\begin{remarkbreak}[König's theorem for kinetic energy]
    Notice that
    %
    \begin{align*}
        T
            &= \frac{1}{2} \sum_{i=1}^n m_i v_i^2
             = \frac{1}{2} \sum_{i=1}^n m_i \norm{\vec{v}_i' + \dot{\vec{R}}}^2 \\
            &= \frac{1}{2} \sum_{i=1}^n m_i (v_i')^2
              + \frac{1}{2} \sum_{i=1}^n m_i \dot{R}^2
              + \dot{\vec{R}} \cdot \sum_{i=1}^n m_i \vec{v}_i'.
    \end{align*}
    %
    But the last term vanishes since $\sum_{i=1}^n m_i \vec{v}_i'$ is the time derivative of $\sum_{i=1}^n m_i \vec{r}_i'$, which is zero. Hence
    %
    \begin{equation*}
        T
            = T' + T_\CM,
    \end{equation*}
    %
    which is \emph{König's theorem} for kinetic energy.
\end{remarkbreak}


\addtocounter{section}{7}
\section{Energy of Interaction of Two Particles}

Consider two particles numbered 1 and 2, and let particle $j$ act on particle $i \neq j$ via a force $\vec{F}_{ij}$. We assume that the force depends only on the position of the two particles, and perhaps time. Focusing on $\vec{F}_{12}$ we thus have e.g.\footnote{We use the physicist's notation to describe the domain of functions; the codomain is either $\reals$ or $\reals^3$, and we distinguish these by denoting vector-valued functions with boldface letters, similar to other vector-valued quantities. Thus the notation $\vec{F}_{12} = \vec{F}_{12}(\vec{r}_1, \vec{r}_2, t)$ means that $\vec{F}_{12}$ is a function $\Omega \to \reals^3$, where $\Omega \subseteq \reals^3 \prod \reals^3 \prod \reals$ is the set of permitted values of $(\vec{r}_1, \vec{r}_2, t)$.} $\vec{F}_{12} = \vec{F}_{12}(\vec{r}_1, \vec{r}_2, t)$. Assuming that the two particles are isolated, we have
%
\begin{equation*}
    \vec{F}_{12}(\vec{r}_1 + \vec{h}, \vec{r}_2 + \vec{h}, t)
    = \vec{F}_{12}(\vec{r}_1, \vec{r}_2, t)
\end{equation*}
%
for all vectors $\vec{h}$, i.e., the force is translation invariant.

Now assume that $\vec{r}_2 = \vec{0}$ at some time\footnote{As far as I can tell, the following arguments do not require that we measure the positions of the particles in an inertial frame, so we may set $\vec{r}_2 = \vec{0}$ at all $t$. In this case we may let the force and the potential be functions of $t$ instead of only being parametrised by $t$.} $t$, which we can always accomplish by changing coordinates. Further assume that the force $\vec{r}_1 \mapsto \vec{F}_{12}(\vec{r}_1, \vec{0}, t)$ is derived from a potential $U_t = U_t(\vec{r}_1)$, depending on $t$. That is, we require that the line integral of the above force between any two points is independent of path, when we keep $t$ fixed. Next, no longer fix particle 2 at the origin. Since the force is translation invariant, it follows that\footnote{By $\grad U_t(\vec{r}_1 - \vec{r}_2)$ below we mean, seemingly contrary to Taylor, the value of the function $\grad U_t$ at the point $\vec{r}_1 - \vec{r}_2$.}
%
\begin{equation*}
    \vec{F}_{12}(\vec{r}_1, \vec{r}_2, t)
        = \vec{F}_{12}(\vec{r}_1 - \vec{r}_2, \vec{0}, t)
        = -\grad U_t(\vec{r}_1 - \vec{r}_2).
\end{equation*}
%
Next define a new potential $U_{12} = U_{12}(\vec{r}_1, \vec{r}_{2}, t) = U_t(\vec{r}_1 - \vec{r}_2)$. Denoting by $\grad_1$ the gradient operator with respect to the first three arguments, i.e. the three coordinates of $\vec{r}_1$, we thus find that
%
\begin{equation*}
    \vec{F}_{12}(\vec{r}_1, \vec{r}_2, t)
        = -\grad_1 U_{12}(\vec{r}_1, \vec{r}_2, t).
\end{equation*}
%
It now follows that
%
\begin{equation*}
    \vec{F}_{21}(\vec{r}_1, \vec{r}_2, t)
        = -\vec{F}_{12}(\vec{r}_1, \vec{r}_2, t)
        = \grad_1 U_{12}(\vec{r}_1, \vec{r}_2, t)
        = -\grad_2 U_{12}(\vec{r}_1, \vec{r}_2, t),
\end{equation*}
%
where the operator $\grad_2$ is defined analogously to $\grad_1$. Hence we indeed have the identities $\vec{F}_{12} = -\grad_1 U_{12}$ and $\vec{F}_{21} = -\grad_2 U_{12}$, so both forces are derived from a single potential energy function.\footnote{Compare Taylor's equation (4.83):
%
\begin{align*}
    \text{(Force on particle 1)} &= -\grad_1 U \\
    \text{(Force on particle 2)} &= -\grad_2 U
\end{align*}
%
This is simply wrong as stated, as Taylor's choice of $U$ requires him to apply e.g. $-\grad_1$ to the function $(\vec{r}_1, \vec{r}_2) \mapsto U(\vec{r}_1 - \vec{r}_2)$, and not to $U$ itself.}


\addtocounter{chapter}{4}
\chapter{Mechanics in Noninertial Frames}

\addtocounter{section}{2}
\section{The Angular Velocity Vector}

\subsection{Isometries}

Let $X$ and $Y$ be normed vector spaces, and let $S \subseteq X$ be a subset. Notice that a (not necessarily linear) map $\phi \colon S \to Y$ is an isometry if
%
\begin{equation*}
    \norm{\phi(\vec{x}) - \phi(\vec{x}')}
        = \norm{\vec{x} - \vec{x}'}
\end{equation*}
%
for all $\vec{x}, \vec{x}' \in S$.

\begin{lemma}
    \label{lem:isometry-preserves-inner-product}
    Let $X,Y$ be normed spaces and $S \subseteq X$, and let $\phi \colon S \to Y$ be a map such that $\phi(\vec{0}) = \vec{0}$. Then $\phi$ is an isometry if and only if
    %
    \begin{equation*}
        \inner{\phi(\vec{x})}{\phi(\vec{x}')}
            = \inner{\vec{x}}{\vec{x}'}
    \end{equation*}
    %
    for all $\vec{x}, \vec{x}' \in S$.
\end{lemma}

\begin{proof}
    The \enquote{if} part is obvious, so we prove the converse. Notice that
    %
    \begin{equation*}
        \norm{\phi(\vec{x})}
            = \norm{\phi(\vec{x}) - \phi(\vec{0})}
            = \norm{\vec{x} - \vec{0}}
            = \norm{\vec{x}}.
    \end{equation*}
    %
    Hence the calculation
    %
    \begin{align*}
        \norm{\phi(\vec{x})}^2 - 2 \inner{\phi(\vec{x})}{\phi(\vec{x}')} + \norm{\phi(\vec{x}')}^2
            &= \norm{\phi(\vec{x}) - \phi(\vec{x}')}^2 \\
            &= \norm{\vec{x} - \vec{x}'}^2 \\
            &= \norm{\vec{x}}^2 - 2 \inner{\vec{x}}{\vec{x}'} + \norm{\vec{x}'}^2
    \end{align*}
    %
    implies that $\inner{\phi(\vec{x})}{\phi(\vec{x}')} = \inner{\vec{x}}{\vec{x}'}$ as desired.
\end{proof}

\begin{proposition}
    \label{prop:isometry-characterisation-Euclidean-space}
    Every isometry $\phi \colon \reals^d \to \reals^d$ is on the form
    %
    \begin{equation*}
        \phi(\vec{x})
            = A \vec{x} + \vec{b}, \quad \vec{x} \in \reals^d,
    \end{equation*}
    %
    for a unique $d \prod d$ matrix $A$ and vector $\vec{b} \in \reals^d$. Furthermore, $A$ is orthogonal.
\end{proposition}
%
In particular, any isometry that preserves the origin is linear.

\begin{proof}
    We first show that $\phi(\vec{x}) = \phi_0(\vec{x}) + \vec{b}$ for some $\vec{b} \in \reals^d$, where $\phi_0 \colon \reals^d \to \reals^d$ is an isometry fixing the origin. Simply let $\vec{b} = \phi(\vec{0})$ and $\phi_0(\vec{x}) = \phi(\vec{x}) - \vec{b}$. Then
    %
    \begin{equation*}
        \phi_0(\vec{0})
            = \phi(\vec{0}) - \vec{b}
            = \vec{b} - \vec{b}
            = \vec{0},
    \end{equation*}
    %
    and for $\vec{x}, \vec{y} \in \reals^d$ we have
    %
    \begin{equation*}
        \norm{ \phi_0(\vec{x}) - \phi_0(\vec{y}) }
            = \norm{ \phi(\vec{x}) - \phi(\vec{y}) }
            = \norm{\vec{x} - \vec{y}},
    \end{equation*}
    %
    so $\phi_0$ is an isometry.

    Next we show that $\phi_0$ is linear; since it is an isometry, this will imply the existence of an orthogonal matrix $A$ as above. For $\vec{x}, \vec{y}, \vec{z} \in \reals^d$ we have by \cref{lem:isometry-preserves-inner-product}
    %
    \begin{align*}
        \inner{ \phi_0(\vec{x} + \vec{y}) }{ \phi_0(\vec{z}) }
            &= \inner{ \vec{x} + \vec{y} }{ \vec{z} } \\
            &= \inner{ \vec{x} }{ \vec{z} } + \inner{ \vec{y} }{ \vec{z} } \\
            &= \inner{ \phi_0(\vec{x}) }{ \phi_0(\vec{z}) } + \inner{ \phi_0(\vec{y}) }{ \phi_0(\vec{z}) } \\
            &= \inner{ \phi_0(\vec{x}) + \phi_0(\vec{y}) }{ \phi_0(\vec{z}) }.
    \end{align*}
    %
    Since $\phi_0$ preserves orthogonality, it maps an orthogonal basis into an orthogonal basis. Replacing $\vec{z}$ by the elements in such a basis, it follows that $\inner{ \phi_0(\vec{x} + \vec{y}) }{ \vec{w} } = \inner{ \phi_0(\vec{x}) + \phi_0(\vec{y}) }{ \vec{w} }$ for all $\vec{w} \in \reals^d$. Hence $\phi_0(\vec{x} + \vec{y}) = \phi_0(\vec{x}) + \phi_0(\vec{y})$. We similarly find that, for $\alpha \in \reals$,
    %
    \begin{align*}
        \inner{ \phi_0(\alpha \vec{x}) }{ \phi_0(\vec{z}) }
            &= \inner{ \alpha \vec{x} }{ \vec{z} } \\
            &= \alpha \inner{ \vec{x} }{ \vec{z} } \\
            &= \alpha \inner{ \phi_0(\vec{x}) }{ \phi_0(\vec{z}) } \\
            &= \inner{ \alpha \phi_0(\vec{x}) }{ \phi_0(\vec{z}) },
    \end{align*}
    %
    implying that $\phi_0(\alpha \vec{x}) = \alpha \phi_0(\vec{x})$ Thus $\phi_0$ is linear.

    To show uniqueness, assume that $A \vec{x} + \vec{b} = \phi(\vec{x}) = A' \vec{x} + \vec{b}'$ for all $\vec{x} \in \reals^d$. Then $\vec{b} = \phi(\vec{0}) = \vec{b}'$, so $A \vec{x} = A' \vec{x}$. Since this holds for all $\vec{x}$, it follows that $A = A'$ as desired.
\end{proof}

\fleuronbreak

Consider a body in space, and let $S(t) \subseteq \reals^3$ denote the points in the body at time $t$. For each $t$ there is a map $\phi_t \colon S(0) \to S(t)$ mapping each point (or particle) in $S(0)$ to the position it inhabits at time $t$. If the body performs rigid motion, then each $\phi_t$ must be an isometry (indeed, we may take this as part of the definition of rigid motion, along with the transformation preserving orientation). However, to apply \cref{prop:isometry-characterisation-Euclidean-space} to $\phi_t$ it must be defined on $\reals^3$. A natural question is then whether it is possible to extend an isometry from a subset of $\reals^3$ to all $\reals^3$. The answer is affirmative:

\begin{lemma}
    Let $S \subseteq \reals^d$ and let $\phi \colon S \to \reals^d$ be an isometry. Then $\phi$ extends to an isometry $\tilde\phi \colon \reals^d \to \reals^d$.
\end{lemma}

\newcommand{\Span}{\operatorname{span}}

\begin{proof}
    By translation we may assume that $\phi(\vec{0}) = \vec{0}$. First extend $\phi$ by linearity to $\Span S$. This is well-defined, since if $\vec{x}_1, \ldots, \vec{x}_n \in S$ and $\alpha_1, \ldots, \alpha_n \in \reals$, then by \cref{lem:isometry-preserves-inner-product} we have
    %
    \begin{equation*}
        \norm[\bigg]{ \sum_{i=1}^n \alpha_i \phi(\vec{x}_i) }^2
            = \sum_{i,j=1}^n \alpha_i \alpha_j \inner{\phi(\vec{x}_i)}{\phi(\vec{x}_j)}
            = \sum_{i,j=1}^n \alpha_i \alpha_j \inner{\vec{x}_i}{\vec{x}_j}
            = \norm[\bigg]{ \sum_{i=1}^n \alpha_i \vec{x}_i }^2.
    \end{equation*}
    %
    This also shows that $\phi$ extends to an isometry on $\Span S$, and in particular $\phi$ is linear by \cref{prop:isometry-characterisation-Euclidean-space}.

    Since $\phi$ is linear it maps $\Span S$ to a subspace $\phi(\Span S)$ of $\reals^d$ with the same dimension. Let $U$ and $V$ be the orthogonal complements of $\Span S$ and $\phi(\Span S)$ respectively, and let $\psi \colon U \to V$ be an isometry. Letting $\tilde{\phi} = \phi \oplus \psi$, it follows easily by Pythagoras' theorem that $\tilde{\phi}$ is also an isometry as desired.
\end{proof}

Next we consider the case where the motion of the body is such that some point is fixed. By choosing coordinates we may assume that this point is the origin, so the motion is given by a linear transformation. Furthermore, for such a motion to be rigid it cannot be a reflection, so we require that it has determinant $1$. In this case the motion turns out to be a \emph{rotation}, i.e. there is an orthonormal basis of $\reals^3$ with respect to which the transformation has the matrix
%
\begin{equation}
    \label{eq:rotation-matrix}
    \begin{pmatrix}
        1 & 0 & 0 \\
        0 & a & -b \\
        0 & b & a
    \end{pmatrix}
\end{equation}
%
for appropriate $a,b \in \reals$ such that $a^2 + b^2 = 1$. That is, $a = \cos\theta$ and $b = \sin\theta$ for some \emph{angle of rotation} $\theta \in [0, 2\pi)$. The span of an eigenvector corresponding to the eigenvalue $1$ is called an \emph{axis of rotation}.

\begin{theorem}[Euler's Theorem]
    Let $\phi \colon \reals^3 \to \reals^3$ be an isometry fixing the origin with determinant $1$. Then $\phi$ is a rotation. If $\phi$ is nontrivial, then it has a unique axis of rotation and a unique angle of rotation.
\end{theorem}

\begin{proof}
    First we show that $\phi$ has $1$ as an eigenvalue. Let $p_\phi(t) = \det(tI - \phi)$ be its characteristic polynomial. The leading term of $p_\phi$ is then $t^3$, so $p_\phi(t) \to \infty$ as $t \to \infty$. On the other hand $p_\phi(0) = -\det\phi = -1$, so $p_\phi$ has a zero on the positive real axis, i.e. $\phi$ has a positive eigenvalue. But this must be $1$ since $\phi$ is an isometry.

    Choose an orthonormal basis for $\reals^3$ whose first element is an eigenvector corresponding to the eigenvalue $1$. Since the matrix of $\phi$ is in this basis is orthogonal, it is precisely on the form \cref{eq:rotation-matrix}, proving the first claim.

    For the uniqueness claims, assume that $\phi$ is nontrivial. Since $\phi$ is orthogonal it is normal, so it has three (complex) eigenvalues, counted with multiplicity. If two eigenvalues are $1$, then the third must also be since $\det\phi = 1$. But then $\phi$ would be trivial, so this is impossible. Hence the eigenspace corresponding to the eigenvalue $1$ is one-dimensional, so $\phi$ has a unique axis of rotation.

    [TODO uniqueness of angle?]
\end{proof}


\subsection{Time derivatives of rotating vectors}

Assume that the body rotates with constant angular velocity $\vec{\omega}$. By changing coordinates we may assume that $\vec{\omega}$ is parallel to the $x$-axis. The corresponding rotation matrix is then
%
\begin{equation*}
    R(t) =
    \begin{pmatrix}
        1 & 0 & 0 \\
        0 & \cos \omega t & -\sin \omega t \\
        0 & \sin \omega t & \cos \omega t
    \end{pmatrix}.
\end{equation*}
%
If $\vec{r}_0 = (r_1, r_2, r_3)$ is any vector \enquote{fixed} in the rotating body, then at time $t$ it will have rotated into the vector $\vec{r}(t) = R(t) \vec{r}_0$. We claim that $\dot{\vec{r}}(t) = \vec{\omega} \prod \vec{r}(t)$. To see this, we simply calculate each side. First of all we have
%
\begin{equation*}
    \dot{\vec{r}}(t)
        = \dot{R}(t) \vec{r}_0
        = -\omega \begin{pmatrix}
            0 & 0 & 0 \\
            0 & \sin \omega t & \cos \omega t \\
            0 & -\cos \omega t & \sin \omega t
        \end{pmatrix} \vec{r}_0
        = -\omega \begin{pmatrix}
            0 \\
            r_2 \sin \omega t + r_3 \cos \omega t \\
            -r_2 \cos \omega t + r_3 \sin \omega t
        \end{pmatrix}.
\end{equation*}
%
Next, since $\vec{\omega} = (\omega,0,0)$ in our chosen coordinates, we have
%
\begin{equation*}
    \vec{\omega} \prod \vec{r}(t)
        = \vec{\omega} \prod R(t) \vec{r}_0
        = \vec{\omega} \prod \begin{pmatrix}
            r_1 \\
            r_2 \cos \omega t - r_3 \sin \omega t \\
            r_2 \sin \omega t + r_3 \cos \omega t
        \end{pmatrix}
        = \omega \begin{pmatrix}
            0 \\
            - r_2 \sin \omega t - r_3 \cos \omega t \\
            r_2 \cos \omega t - r_3 \sin \omega t
        \end{pmatrix}.
\end{equation*}
%
The claim thus follows.


\section{Time Derivatives in a Rotating Frame}

\newcommand{\calS}{\mathcal{S}}

Let $\calS_0$ be an inertial frame and $\calS$ a noninertial frame with the same origin. Assume that $\calS$ is rotating around its origin with angular velocity $\vec{\Omega}$, and let $R(t)$ be the corresponding family of rotation matrices. That is, if $\vec{x}$ is a vector that is fixed in $\calS$, then measured from $\calS_0$ it is $R(t) \vec{x}$.

Let $\calE = (\vec{e}_1, \vec{e}_2, \vec{e}_3)$ be an orthonormal basis; we think of these vectors as fixed in $\calS_0$. Furthermore let $\vec{e}_i'(t) = R(t)\inv \vec{e}_i$, so that $\calE'(t) = (\vec{e}_1'(t), \vec{e}_2'(t), \vec{e}_3'(t))$ is an orthonormal basis for each $t$, fixed in $\calS$. Hence if $\vec{x} = (x_1, x_2, x_3)$ is a coordinate vector with respect to $\calE$, we have
%
\begin{equation*}
    R(t)\inv \vec{x} = \sum_{i=1}^3 x_i R(t)\inv \vec{e}_i
        = \sum_{i=1}^3 x_i R(t)\inv \vec{e}_i
\end{equation*}

That is, we consider a time-dependent orthonormal basis $(\vec{e}_1(t), \vec{e}_2(t), \vec{e}_3(t))$ that is at rest in $\calS$. If $\vec{x}(t)$ is a time-dependent vector as measured in $\calS_0$, we denote by $\vec{x}'(t)$ the same vector as measured in $\calS$, i.e. the coordinate vector of $\vec{x}(t)$ with respect to $(\vec{e}_1(t), \vec{e}_2(t), \vec{e}_3(t))$.


\end{document}