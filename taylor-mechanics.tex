% Document setup
\documentclass[article, a4paper, 11pt, oneside]{memoir}
\usepackage[utf8]{inputenc}
\usepackage[T1]{fontenc}
\usepackage[UKenglish]{babel}

% Document info
\newcommand\doctitle{Taylor, \emph{Classical Mechanics}}
\newcommand\docauthor{Danny Nygård Hansen}

% Formatting and layout
\usepackage[autostyle]{csquotes}
\usepackage[final]{microtype}
\usepackage{xcolor}
\frenchspacing
\usepackage{latex-sty/articlepagestyle}
\usepackage{latex-sty/articlesectionstyle}

% Fonts
\usepackage{amssymb}
\usepackage[largesmallcaps,partialup]{kpfonts}
\DeclareSymbolFontAlphabet{\mathrm}{operators} % https://tex.stackexchange.com/questions/40874/kpfonts-siunitx-and-math-alphabets
\linespread{1.06}
% \let\mathfrak\undefined
% \usepackage{eufrak}
\DeclareMathAlphabet\mathfrak{U}{euf}{m}{n}
\SetMathAlphabet\mathfrak{bold}{U}{euf}{b}{n}
% https://tex.stackexchange.com/questions/13815/kpfonts-with-eufrak
\usepackage{inconsolata}

% Hyperlinks
\usepackage{hyperref}
\definecolor{linkcolor}{HTML}{4f4fa3}
\hypersetup{%
	pdftitle=\doctitle,
	pdfauthor=\docauthor,
	colorlinks,
	linkcolor=linkcolor,
	citecolor=linkcolor,
	urlcolor=linkcolor,
	bookmarksnumbered=true
}

% Equation numbering
\numberwithin{equation}{chapter}

% Footnotes
\footmarkstyle{\textsuperscript{#1}\hspace{0.25em}}

% Mathematics
\usepackage{latex-sty/basicmathcommands}
\usepackage{latex-sty/framedtheorems}

% Lists
\usepackage{enumitem}
\setenumerate[0]{label=\normalfont(\arabic*)}

% Bibliography
\usepackage[backend=biber, style=authoryear, maxcitenames=2, useprefix]{biblatex}
\addbibresource{references.bib}

% Title
\title{\doctitle}
\author{\docauthor}

\newcommand{\setF}{\mathbb{F}}
\newcommand{\ev}{\mathrm{ev}}
\newcommand{\calT}{\mathcal{T}}
\newcommand{\calU}{\mathcal{U}}
\newcommand{\calB}{\mathcal{B}}
\newcommand{\calE}{\mathcal{E}}
\newcommand{\calC}{\mathcal{C}}
\newcommand{\calD}{\mathcal{D}}
\newcommand{\calF}{\mathcal{F}}
\newcommand{\calG}{\mathcal{G}}
\newcommand{\calM}{\mathcal{M}}
\newcommand{\calA}{\mathcal{A}}
\newcommand{\calP}{\mathcal{P}}
\newcommand{\calR}{\mathcal{R}}
\newcommand{\calL}{\mathcal{L}}
\newcommand{\calH}{\mathcal{H}}
\newcommand{\borel}{\mathcal{B}}
\newcommand{\measurable}{\mathcal{M}}
\newcommand{\wto}{\Rightarrow}
\DeclarePairedDelimiter{\net}{\langle}{\rangle}
\newcommand{\strucS}{\mathfrak{S}}
\DeclarePairedDelimiter{\gen}{\langle}{\rangle} % Generating set
\newcommand{\frakL}{\mathfrak{L}}


% Physics commands

\DeclarePairedDelimiter{\ket}{\lvert}{\rangle}
% \renewcommand{\vec}{\mathbf}
\newcommand{\grad}{\nabla}


\begin{document}

\maketitle


\addtocounter{chapter}{2}
\chapter{Momentum and Angular Momentum}

\addtocounter{section}{2}
\section{The Centre of Mass}

\begin{remarkbreak}[The centre of mass and convexity]
    Consider collections of particles at positions $\vec{r}_{11}, \ldots, \vec{r}_{1k}$ and $\vec{r}_{21}, \ldots, \vec{r}_{2l}$, with masses $m_{11}, \ldots, m_{1k}$ and $m_{21}, \ldots, m_{2l}$, respectively. Let $M_1 = \sum_{i=1}^k m_{1i}$ and $M_2 = \sum_{j=1}^l m_{2j}$ be the total masses of the two collections of particles, and let
    %
    \begin{equation*}
        \vec{R}_1
            = \frac{1}{M_1} \sum_{i=1}^k m_{1i} \vec{r}_{1i}
        \quad \text{and} \quad
        \vec{R}_2
            = \frac{1}{M_2} \sum_{j=1}^l m_{2j} \vec{r}_{2j}
    \end{equation*}
    %
    be their centres of mass. The centre of mass of the whole system is then
    %
    \begin{align*}
        \vec{R}
            &= \frac{1}{M_1 + M_2} \biggl( \sum_{i=1}^k m_{1i} \vec{r}_{1i} + \sum_{j=1}^l m_{2j} \vec{r}_{2j} \biggr) \\
            &= \frac{1}{M_1 + M_2} \biggl( \frac{M_1}{M_1} \sum_{i=1}^k m_{1i} \vec{r}_{1i} + \frac{M_2}{M_2} \sum_{j=1}^l m_{2j} \vec{r}_{2j} \biggr) \\
            &= \frac{M_1 \vec{R}_1 + M_2 \vec{R}_2}{M_1 + M_2}.
    \end{align*}
    %
    This obviously extends to any finite number of collections of particles.

    In particular, it follows that the centre of mass of a finite collection of particles lies in the convex hull of their position vectors. For the above shows that $\vec{R}$ is a convex combination of $\vec{R}_1$ and $\vec{R}_2$, so this follows by induction on the number of particles. In fact, this holds for any weighted average (with non-negative weights) of points in any dimension.
\end{remarkbreak}


\section{Ang}

[TODO]

\section{Angular Momentum for Several Particles}

\newcommand{\CM}{\mathrm{CM}}

\begin{remarkbreak}[König's theorem for angular momenta]
    For $i = 1, \ldots, n$ consider a particle with mass $m_i$ and position vector $\vec{r}_i$ with respect to an origin $O$ at rest in an inertial frame. Furthermore, let $\vec{p}_i$ be its momentum. If $\vec{R}$ is the centre of mass of all $n$ particles, then we let $\vec{r}_i' = \vec{r} - \vec{R}$ be the relative position of the $i$th particle, and let $\vec{p}_i' = m_i \dot{\vec{r}}_i'$. Denote by $M$ the total mass of the particles, and let $\vec{P}$ be the total momentum (equivalently $\vec{P} = M \dot{\vec{R}}$). The angular momentum of the $i$th particle with respect to the centre of mass is then
    %
    \begin{align*}
        \vec{l}_i'
            &= \vec{r}_i' \prod \vec{p}_i'
             = (\vec{r}_i - \vec{R}) \prod (\vec{p}_i - m_i \dot{\vec{R}}) \\
            &= \vec{l}_i - m_i \vec{r}_i \prod \dot{\vec{R}} - \vec{R} \prod \vec{p}_i + m_i \vec{R} \prod \dot{\vec{R}}.
    \end{align*}
    %
    Therefore, the total angular momentum with respect to the centre of mass is
    %
    \begin{align*}
        \vec{L}'
            &= \sum_{i=1}^n \vec{l}_i'
             = \vec{L} - M \vec{R} \prod \dot{\vec{R}} - \vec{R} \prod \vec{P} + M \vec{R} \prod \dot{\vec{R}} \\
            &= \vec{L} - \vec{R} \prod \vec{P}
             = \vec{L} - \vec{L}_\CM,
    \end{align*}
    %
    where $\vec{L}_\CM = \vec{R} \prod \vec{P}$ is the angular momentum of the centre of mass. Hence
    %
    \begin{equation*}
        \vec{L} = \vec{L}' + \vec{L}_\CM,
    \end{equation*}
    %
    which is \emph{König's theorem} for angular momenta.
\end{remarkbreak}


\begin{remarkbreak}[Decomposition of torque]
    Next let each particle $i$ be acted upon by an external force $\vec{F}_i$, and let $\vec{F}$ be the total force on the system. If the total external torque relative to the origin $O$ is $\vec{\Gamma}$, then the external torque relative to the centre of mass is
    %
    \begin{align*}
        \vec{\Gamma}'
            &= \sum_{i=1}^n \vec{r}_i' \prod \vec{F}_i
             = \sum_{i=1}^n (\vec{r}_i - \vec{R}) \prod \vec{F}_i \\
            &= \sum_{i=1}^n \vec{\Gamma}_i - \vec{R} \prod \vec{F}
             = \vec{\Gamma} - \vec{\Gamma}_\CM,
    \end{align*}
    %
    where $\vec{\Gamma}_\CM = \vec{R} \prod \vec{F}$ is the torque on the centre of mass. That is,
    %
    \begin{equation*}
        \vec{\Gamma}
            = \vec{\Gamma}' + \vec{\Gamma}_\CM.
    \end{equation*}
    %
    Now notice that 
    %
    \begin{equation*}
        \dot{\vec{L}}_\CM
            = (\dot{\vec{R}} \prod \vec{P}) + (\vec{R} \prod \dot{\vec{P}})
            = \vec{R} \prod \vec{F}
            = \vec{\Gamma}_\CM,
    \end{equation*}
    %
    since $\dot{\vec{R}}$ and $\vec{P} = M\dot{\vec{R}}$ are parallel. König's theorem thus implies that
    %
    \begin{equation*}
        \dot{\vec{L}}'
            = \dot{\vec{L}} - \dot{\vec{L}}_\CM
            = \vec{\Gamma} - \vec{\Gamma}_\CM
            = \vec{\Gamma}'.
    \end{equation*}
    %
    That is, relative to the centre of mass, the external torque is the time derivative of the angular momentum, even if the centre of mass frame is not inertial.
\end{remarkbreak}


\chapter{Energy}

\addtocounter{section}{8}

\section{Energy of Interaction of Two Particles}

Consider two particles numbered 1 and 2, and let particle $i$ act on particle $j \neq i$ via a force $\vec{F}_{ij}$. We assume that the force depends only on the position of the two particles, and perhaps time. Focusing on $\vec{F}_{12}$ we thus have e.g.\footnote{We use the physicist's notation to describe the domain of functions; the codomain is either $\reals$ or $\reals^3$, and we distinguish these by denoting vector-valued functions with boldface letters, similar to other vector-valued quantities. Thus the notation $\vec{F}_{12} = \vec{F}_{12}(\vec{r}_1, \vec{r}_2, t)$ means that $\vec{F}_{12}$ is a function $\Omega \to \reals^3$, where $\Omega \subseteq \reals^3 \prod \reals^3 \prod \reals$ is the set of permitted values of $(\vec{r}_1, \vec{r}_2, t)$.} $\vec{F}_{12} = \vec{F}_{12}(\vec{r}_1, \vec{r}_2, t)$. Assuming that the two particles are isolated, we have
%
\begin{equation*}
    \vec{F}_{12}(\vec{r}_1 + \vec{h}, \vec{r}_2 + \vec{h}, t)
    = \vec{F}_{12}(\vec{r}_1, \vec{r}_2, t)
\end{equation*}
%
for all vectors $\vec{h}$, i.e., the force is translation invariant.

Now assume that $\vec{r}_2 = \vec{0}$ at some time\footnote{As far as I can tell, the following arguments do not require that we measure the positions of the particles in an inertial frame, so we may set $\vec{r}_2 = \vec{0}$ at all $t$.} $t$, which we can always accomplish by changing coordinates. Further assume that the force $(\vec{r}_1, t) \mapsto \vec{F}_{12}(\vec{r}_1, \vec{0}, t)$ is derived from a potential $U_t = U_t(\vec{r}_1)$, parametrised by $t$. That is, we require that the line integral of the above force between any two points is independent of path, when we keep $t$ fixed. Next, no longer fix particle 2 at the origin. Since the force is translation invariant, it follows that\footnote{By $\grad U_t(\vec{r}_1 - \vec{r}_2)$ below we mean, seemingly contrary to Taylor, the value of the function $\grad U_t$ at the point $\vec{r}_1 - \vec{r}_2$.}
%
\begin{equation*}
    \vec{F}_{12}(\vec{r}_1, \vec{r}_2, t)
        = \vec{F}_{12}(\vec{r}_1 - \vec{r}_2, t)
        = -\grad U_t(\vec{r}_1 - \vec{r}_2).
\end{equation*}
%
Next define a new potential $U_{12} = U_{12}(\vec{r}_1, \vec{r}_{2}, t) = U_t(\vec{r}_1 - \vec{r}_2)$. Denoting by $\grad_1$ the gradient operator with respect to the first three arguments, i.e. the three coordinates of $\vec{r}_1$, we thus find that
%
\begin{equation*}
    \vec{F}_{12}(\vec{r}_1, \vec{r}_2, t)
        = -\grad_1 U_{12}(\vec{r}_1, \vec{r}_2, t).
\end{equation*}
%
We similarly find that
%
\begin{equation*}
    \vec{F}_{21}(\vec{r}_1, \vec{r}_2, t)
        = -\vec{F}_{12}(\vec{r}_1, \vec{r}_2, t)
        = \grad_1 U_{12}(\vec{r}_1, \vec{r}_2, t)
        = -\grad_2 U_{12}(\vec{r}_1, \vec{r}_2, t),
\end{equation*}
%
where the operator $\grad_2$ is defined analogously to $\grad_1$.

\end{document}